\documentclass[conference]{IEEEtran}
\IEEEoverridecommandlockouts
% The preceding line is only needed to identify funding in the first footnote. If that is unneeded, please comment it out.
\usepackage{cite}
\usepackage{amsmath,amssymb,amsfonts}
\usepackage{algorithmic}
\usepackage[tight,footnotesize]{subfigure}
\usepackage{graphicx}
\usepackage{textcomp}
\usepackage{xcolor}
\usepackage[ruled,vlined]{algorithm2e}


\def\BibTeX{{\rm B\kern-.05em{\sc i\kern-.025em b}\kern-.08em
    T\kern-.1667em\lower.7ex\hbox{E}\kern-.125emX}}
\begin{document}

\title{Future Generation Multi-tier Video Streaming Networks
}

\author{\IEEEauthorblockN{1\textsuperscript{st} Given Name Surname}
\IEEEauthorblockA{\textit{dept. name of organization (of Aff.)} \\
\textit{name of organization (of Aff.)}\\
City, Country \\
email address}
\and
\IEEEauthorblockN{2\textsuperscript{nd} Given Name Surname}
\IEEEauthorblockA{\textit{dept. name of organization (of Aff.)} \\
\textit{name of organization (of Aff.)}\\
City, Country \\
email address}
\and
\IEEEauthorblockN{3\textsuperscript{rd} Given Name Surname}
\IEEEauthorblockA{\textit{dept. name of organization (of Aff.)} \\
\textit{name of organization (of Aff.)}\\
City, Country \\
email address}
}

\maketitle

\begin{abstract}
Video streaming services have been experiencing changes in the recent years, where the number of users consuming a variety of video streaming has been increasing exponentially in Video-on-Demand~(VoD) platforms. In order to accommodate the growing video traffic demand, new solutions have to be designed for the Future of the Internet to provide good users' Quality of Experience~(QoE). In this way, this research project proposes to evaluate the impact on the performance of VoD introduced by new distributed computing infrastructures in edge/cloud computing scenarios.

No entanto, um serviço de nuvem centralizado apresenta alguns problemas, como maior latência e congestionamento da núcleo da rede. Portanto, para melhorar os serviços de vı́deo, é de suma importância distribuir adequadamente os fluxos de vı́deo de acordo com seus requisitos: uma infraestrutura de vı́deos em tempo real em nu- vem é um serviço interativo que precisa de atrasos reduzidos (alguns milissegundos), enquanto uma entrega de VoD não interativa pode tolerar maior atraso sem prejudicar a qualidade da experiência. Um gerenciamento e orquestração adequados da entrega de video pela Internet é essencial para a coexistência suave de serviços de vı́deo heterogêneos. Este projeto propõe o uso da hierarquia de névoa/nuvem para projetar um streaming de video DASH cooperativo em Cidades Inteligentes, implantando um conjunto de mecanismos capaz de oferecer uma Qualidade de Experiência (QoE) aprimorada para usuários finais.
\end{abstract}

\begin{IEEEkeywords}
component, formatting, style, styling, insert
\end{IEEEkeywords}


\section{Introduction}
\label{sec:introduction}

{\color{blue} In this section we will contextualize video conference technology and scalable problems, emphasizing the need of create a hierarchical topology to , and describe the content of each section of the paper.}

% Introduction of the Content and MEC service provider  
Over the years, the Internet traffic has growing exponentially around the world. Mainly, due to the multimedia content streaming which, currently, represents the 70\% of the whole traffic. To delivery a video, the streaming is generated by Video-on-Demand services and distributed by the the Over-The-top providers, while they attempt to satisfy the quality of experience~(QoE) to a wide range of users subscribers. To provide the best user satisfaction, OTT providers may take advantages of the high data-rate and low-latency in the edge network. Thus, caching the video closer to the end-user and impacting positively in the QoE degree performance.

% Challenges
The advantages of using the edge of the network to store the video helps in user satisfaction but also impact. However, there are certain precautions when choosing the nodes for caching the video that need to be taken into account. 

% Objective
Motivated by the advantages of multi-level fog/cloud scenarios to improve user satisfaction, as well as video traffic grows exponentially. This article presents the need to have an orchestrator for the provision of video streaming content. We discuss the impact of a 

% Organization paper

\section{Related Work}
\label{sec:related-work}

This section describes the related works in edge-cloud computing for video streaming. here, some representative works in QoE are summarized, regarding the strategies proposals to improve the video provisioning by the edge.

In regard of Dynamic Adaptive Streaming over HTTP~(DASH) a number of works take in consideration   

\section{Analysis and Opportunities of a Video Streaming Architecture}
\label{sec:system-archi}

%{\color{blue} This section explains the architecture system, and what it is  for, explaining some protocols flowcharts. Studying the protocols and focusing in the system part of the experiments.}

%This section explains the modeling used in our experiments. We first provide an overview of the proposed video streaming service impacts in multi-tier networks. The remaining subsections describe the implementation of this service architecture.
Designing a cache hierarchy on vertically organized edge nodes with an arbitrary number of tiers can present improvements in users’ QoE~\cite{rana2018vertical}. The aforementioned architecture works toward such advantages by serving the requested content as close as possible to the end-user, efficiently forwarding requests between parent edge nodes within the hierarchy and balancing the video traffic in terms of hop counts and users attended. In addition, the network core congestion is reduced since it represents an operational overhead for the content provider.
As a preliminary outcome achieved by a multi-tier network experiment, the QoE impact over a video streaming service is assessed. Thereafter, we describe some results about the QoE characteristics and insights on the opportunities of caching multimedia content in edge nodes of multi-tier networks.

\subsection{Impact of Fog Multi-tier Network Approach}

\begin{figure*}
    \centering
    \includegraphics[width=\linewidth]{images/qoe-multi-level-3.pdf}
    \caption{The number of bitrate switches, stalls, buffer size, the startup delay in seconds of a DASH player requesting a video with 10 bitrate levels varying from 50 to 4,500Kbps and from nodes in different tiers.}
    \label{fig:impact-two-layers}
\end{figure*}

To illustrate the differences in users' performance requesting a video from cache nodes in different tiers, consider two users requesting the same multimedia content from different layers. Then, and analysis of the impacts over the network is performed.  Figure~\ref{fig:impact-two-layers} illustrates a two-tier network model. The graphs show the results of bitrate, interruptions (stalls), video buffer, and representations switch, respectively, from left to right.%and along x-axis the simulation time. 
Although an user requests a cache edge node video, it can be located in layers L1 or L2. Each layer level has different network factors, e.g., load, latency, so on. In Figure~\ref{fig:multi-tier-network}, an L1 layer can be interpreted as a personal computer, Access Point or Base Station, and this layer transmits the video content through wired or wireless communication channels, whereas in the L2 layer, a specific Edge gateway can be distributed on local edge nodes.

Apart from the initial interruption before the start of the video, both users have had no interruptions during the video execution. However, the user who received the video from the nearest edge layer had a higher bit rate than the user receiving the video from the upmost layer.
Note that user receiving the video from the closest layer have filled the buffer faster, and thus he/she managed to keep the video playing at the best possible resolution.
On the other hand, the user who received the video from a more distant layer worked with the buffer at the limit and had to constantly switch resolutions so that there is no interruption during the video execution. Even without any interruptions in both video playbacks, the transmission from different tiers directly impacts the users' QoE.


\subsection{Multi-tier Edge-Cloud Network Opportunities}

%Esta seção apresenta algumas oportunidades dentro de redes multi-tier edge/cloud para o provisionamento da transmissção de videos. Aproveitar nós próximos aos usuários podem melhorar o funcionamento do rede como um todo, aqui nós discutimos alguns insights que podem ser usados em favor dos stakeholders que utilizam a infraestrutura para transmissão de video. 
Opportunities for resource management in multi-tier edge/cloud networks for the provision of video transmission are discussed below. The advantages of nodes closer to end-users can improve the functioning of the network as a whole. We discuss some insights that can be used in favor of network and provider admin in the infrastructure for video transmission.

\subsubsection{Improving User QoE}

%the cloud distributes the video content to the different levels of the edge. Depending the level which the video is cached the users' experience changes. The architecture is based on held the video distribution with QoE support. The work divides the edge into three layers, to guarantee coverage, storage, upload and download capacity. 
In a multi-tier environment composed of more than one domain of devices, the domain resources can host the video content near the end-users, reducing latency and mitigating the load on the network core. The edge nodes are composed of specific resources combined to carry out the video transmission to integrate video streaming services in such communication environments. Within this context, mechanisms for integrating schemes and video streaming raise as an opportunity to improve QoE.% at each network level.
The results reported in Fig~\ref{fig:impact-two-layers} suggests that it is possible to improve the users' satisfaction using the edge multi-tier network. Depending on the level of allocation of the video service, the video smoothing variation between characteristics of the player changes.


\subsubsection{Potential Bandwidth Saving}

%Video Caching, Analytics, and Delivery at the Wireless Edge: A Survey and Future Directions
Videos streamed in higher quality increase network bandwidth use. Consequently, provisioning from the Cloud will incur high communication expenses. 
The process of delivering part of the video along the network can significantly save bandwidth instead of sending the entire video to an edge server or by lowering the encoding quality of uninteresting portions of the video. Different delivery approaches can have different performances to reduce the uplink bandwidth use. Moreover, none of the surveyed articles have considered the end-to-end design of video streaming, wherein the edge server adapts the video streams based on uplink and downlink bandwidth capacities. Additionally, new forms of video content are being generated today and may present opportunities for bandwidth saving and video services orchestration in edge-cloud infrastructures.


\subsubsection{Cacheability}

%Caching audio/video during peak hours ...
%Manage the QoE users refer to those services where the Controller can centrally control the satisfaction guarantees. The Controller can address this problem by creating a control channel to managed-quality video streaming services over the edge-cloud network.

Nodes located at the edge are responsible for providing resources to VoD providers to allocate their caches.  Note that the caches are to multi-hop away from the content provider/consumer. 
With new possibilities being created to offer better services that work on the internet. As the problem becomes complex and new challenges arise to be solved. Different characteristics have to be studied in multi-tier environments into the edge, such as cache allocation, placement, replacement, and selection caches, usually in real-time decision-making processes. In addition, content can be dynamically split into a set of pieces to serve the end-users, taking into account different characteristics, such as the mobility presented by the user and the possibility of predicting the movement direction. 



\section{Performance Evaluation}
\label{sec:results}

In this section we describe the experimental evaluation of the strategy proposed to redirect the users, including the scenarios, metrics, methodology and outcomes.


\subsection{QoE Metric evaluation}

There exist many viewer QoE models in the literature. we will describe the QoE metrics used to score the user satisfaction. Firstly, We compute each video quality chunk by a logarithmic law over bit-rates. Study in~ propose a video quality model for DASH as shown in Equation (1). Each video has $N$ chunks and is encoded with $L$ bitrate levels. $r_i$ represents a specific bitrate level. At each step $i$, the quality of chunk $i$ which is encoded at $l_i$ is defined as:

$$
q(r_i) = a_1 * log(a_2 * (r_i/ r_{|L|}))
$$

To quantify a long-term users' QoE, we require a flexible QoE model that includes the most effective metrics. 
We consider the Eqn. (7) which consists of four metrics: (a) the average chunk perceptual quality, (b) the average number of quality oscillations, (c) the average number of stall events and their durations, and (d) the startup delay. K

\begin{equation}\label{qoe-equation}
QoE_i = \alpha_1 + \alpha_2 + \alpha_3 + \alpha_4
\end{equation}


The QoE has a range of 1 to 5. Where the values 1 = bad, 2 = poor, 3 = fair, 4 = good, and 5 = excellent.
 
 
\subsection{Methodology}

We present three approaches to address the methodological problems identified in a edge/cloud multi-tier network for watching a video streaming: cloud-only 

\subsection{Experimental Setup}

%[29] Christian Kreuzberger, Daniel Posch, and Hermann Hellwagner. Amust framework - adaptive multimedia streaming simulation framework for ns-3 and ndnsim, 2016.
%[30] C. Mueller, S. Lederer, J. Poecher, and C. Timmerer. Demo paper: Libdash - an open source software library for the mpeg-dash standard. In 2013 IEEE International Conference on Multimedia and Expo Workshops (ICMEW), pages 1–2, July 2013.
% Per-title encode optimization, 2015. ; accessed 20-novembro-2019.

To implement DASH servers and users that allow adaptive video streaming, we use Adaptive Multimedia Streaming~(AMuSt)~[1]. The AMuSt framework provides a set of applications for producing and consuming adaptable video, based on the DASH standard. DASH functionality is provided by the libdash library~[2], an open source library that provides an interface to the DASH standard. Currently, libdash is the official reference software for the DASH standard. We consider that users are interested in an available video with ten different bit rate representations \{235kbps, 375kbps, 560kbps, 560kbps, 750kbps, 1050kbps, 1750kbps, 2350kbps, 3000kbps, 4300kbps, 5800kbps\}, which are used by Netflix subsets~[3], which are used by Netflix. [31]. Each representation is divided into 2-second segments. Each experiment are executed once with a video of 1600 seconds~(800 segments). 
For sake of simplicity, the multimedia content used in simulation are already deployed in the edge peering nodes. 
%Figure 3 shows the topology with three levels used in the experiments, each level \{1, 2 and 3\} is represented as a scenario.

%Dash server and the clients allowed to request a multimedia content in DASH format, was used the Adaptive Multimedia Streaming~(AMuSt). The AMust system allow a set o apps to produce and consume the adaptive video, based on Dash pattern. The DASH functionality is provided by the libdash library, an open source library that provides an interface to the DASH standard. Currently, libdash is the reference software official DASH standard.

We simulate the scenario in a binary tree topology with seven nodes and a Cloud Provider connected to the root node of the Binary tree. Where the last four nodes are Access Points~(AP), and the others are edge peering points. Figure~\ref{fig:exp-setup-scenario} illustrate the binary tree scenario.

The AP nodes is implemented on an wireless device which communicates via IEEE 802.11g in 2.4GHz frequency. The APs are connected to the edge peering points by wire and the end-users via wireless. Each user connected to the AP is located exatcly 8 meters away. The Bandwidth available in each link can be seemed in Fig.~\ref{fig:exp-setup-scenario}.

We present three approaches to address the impacts identified in Section~\ref{sec:system-archi} into the edge/cloud multi-tier network: the \textit{cloud-only}, \textit{1\&2 nodes} and mobile-based scenarios. The cloud-only scenario uses just the Cloud Provider node to delivery the video content. \textit{1\&2 nodes} uses the nodes 1 ad 2 as auxialiary nodes to delivery the video. The simulation starts with the users requesting the video from the cloud. When some link detect a congested link  the edge cache abaixo do link is turned on. Thereafter, the users que estão recebendo o video ao longo do link congestionado, são redirecionados para o nó da borda. To moastrar os problemas que podem aparecer em um cenário onde os usuários, e ocorre uma mudança de conexão entre entre os APs. Nós executamos novamente o experimento mantendo as conexões entre os usuários e edge nodes, mas mudamos a conexão entre os usuários e os APs. 

Para mostrar o impacto que pode ocorrer em um cenário sem um gerenciamento adequado das conexões ativas na borda.


\begin{figure}
    \centering
    \includegraphics[width=0.9\linewidth]{images/exp-setup-scenario.pdf}
    \caption{A General Overview of the multi-tier network environment.}
    \label{fig:exp-setup-scenario}
\end{figure}

%To illustrate the idea, we assume tree topology. According to the guideline, a mobile backhaul network is modeled as a two-level hierarchical network. Wireless base stations are connected to aggregation nodes, e.g., service/packet data network gateway (S/P-GW) nodes. Furthermore, aggregation nodes are connected to core nodes, e.g., central office nodes. To the best of our knowledge, separation of traffic from one wireless base stations to multiple aggregation nodes, and from one aggregation node to multiple core nodes is not dominant in current mobile backhaul networks. Thus, we assume a tree-topology backhaul network



% \begin{figure*}
%     \centering
%     \subfigure[]{
%     \includegraphics[width=0.45\linewidth]{images/QoECompare.png}
%     \label{fig:red-comparison-plot}
%     }
%     \subfigure[]{
%     \includegraphics[width=0.45\linewidth]{images/QoEBoxplot.png}
%     \label{fig:co-comparison-boxplot}
%     }

%     \subfigure[]{
%     \includegraphics[width=0.45\linewidth]{images/QoECompare.png}
%     \label{fig:red-comparison-plot}
%     }
%     \subfigure[]{
%     \includegraphics[width=0.45\linewidth]{images/QoEBoxplot.png}
%     \label{fig:red-comparison-boxplot}
%     }

%     \caption{Impact of system on the network performance. Distance \textit{d} between sensor node and antennas of 8m in a semi-NLOS scenario.}
%     \label{fig:comparison-rof-2}
% \end{figure*}

\begin{figure*}
    \centering
    \subfigure[]{
    \includegraphics[width=0.31\linewidth]{images/QoEBoxplot-15u.png}
    \label{fig:red-comparison-plot}
    }
    \subfigure[]{
    \includegraphics[width=0.31\linewidth]{images/QoEBoxplot-20u.png}
    \label{fig:co-comparison-boxplot}
    }
    \subfigure[]{
    \includegraphics[width=0.31\linewidth]{images/QoEBoxplot-25u.png}
    \label{fig:red-comparison-plot}
    }
    
    \caption{Average QoE results for scenarios with 15, 20 and 25 users per AP.}
    \label{fig:comparison-rof-2}
\end{figure*}


\begin{figure*}
    \centering
    \subfigure[]{
    \includegraphics[width=0.31\linewidth]{images/cloud_QoExStartTime15.png}
    \label{fig:rssi-comparison-2}
    }
    \subfigure[]{
    \includegraphics[width=0.31\linewidth]{images/cloud_QoExStartTime20.png}
    \label{fig:plr-comparison-2}
    }
    \subfigure[]{
    \includegraphics[width=0.31\linewidth]{images/cloud_QoExStartTime25.png}
    \label{fig:plr-comparison-2}
    }
    
    \subfigure[]{
    \includegraphics[width=0.31\linewidth]{images/Redicrect_QoExStartTime15.png}
    \label{fig:rssi-comparison-2}
    }
    \subfigure[]{
    \includegraphics[width=0.31\linewidth]{images/Redicrect_QoExStartTime20.png}
    \label{fig:plr-comparison-2}
    }
    \subfigure[]{
    \includegraphics[width=0.31\linewidth]{images/Redicrect_QoExStartTime25.png}
    \label{fig:plr-comparison-2}
    }
    
    \caption{Impact of system on the network performance. Distance \textit{d} between sensor node and antennas of 8m in a semi-NLOS scenario.}
    \label{fig:comparison-qoe-2}
\end{figure*}

\subsection{Results}

We made the experiment illustrated in Figure~\ref{fig:red-comparison-plot} to show the average QoE as shown in~\ref{qoe-equation} to 15, 20 and 25 users. In this experiment, as users reach the final QoE, it tends to be a little lower than the previous one. When looking at the final QoE delta between users $u_{i}$ and $u_{i + 1}$ seem to be irrelevant, but as we increase the delta the QoE starts to become considerable. Through the figure~\ref{fig:co-comparison-boxplot} we can also affirm this behavior, as the number of users increases, the average QoE decreases and the variation between users increases. Also note that the simulation with 25 users per system the system already shows a degradation in the quality perceived by the end user in a negative way.

While Figures~\ref{fig:red-comparison-plot} and~\ref{fig:red-comparison-boxplot} show that a simple video allocation strategy at the edge can help improve the quality of the user experience.

The Figures~\ref{fig:comparison-qoe-2} shows a simple strategy of moving the video to the edge can significantly improve the user's QoE. In this way, the video transmission system is able to provide user satisfaction qualities as well as they tend to keep them watching the video until the end. 

It is important to note that, although the bit rates between the ABR mechanisms are similar, the interruptions they can generate are significantly different, which directly impacts users' QoE. Another point to be noted is the average bit rate between different levels. In the scenario with 20 users, level 3 was able to achieve what is necessary to obtain the highest bit rate representation, thus, network operators can seek to deploy caches during peak hours to provide the best user experience.

\section{Conclusion}
\label{sec:conclusion}

Through a literature review, it can be seen that there are many proposals in the area of ​​adaptive video streaming that do not take into account essential aspects of the user's QoE discussed in this work. In addition, current work on DASH architectures for future generations of smart grids ignores the behavior of the video player in Smart City environments. In the continuation of this work, we intend to work on the implementation of mechanisms to improve the provisioning of video streaming in multilevel architectures. On the client side, aspects of the way in which the video player behaves to provide a delicate balance between cost and customer satisfaction (in terms of QoE), as well as the impact that the insertion of cross traffic can have on the viewer . While on the server side, new simulation scenarios with different domains between nodes in the mist need to be considered, as well as performing the communication between these domains in multilevel environments.

\section*{Acknowledgment}

This work was partially supported by the grant \# 2018/02204-6, São Paulo Research Foundation (FAPESP) and by the European Commission H2020, \# 688941 (FUTEBOL), as well from the Brazilian MCTIC through RNP and CTIC.

\begin{small}
    \bibliographystyle{unsrt}
    \bibliography{references}
\end{small}

\vspace{12pt}

\end{document}
