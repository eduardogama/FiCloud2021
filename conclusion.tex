\section{Conclusion}
\label{sec:conclusion}

%Through a controlable scenario, it can be seen that there are many proposals in the area of ​​adaptive video streaming that do not take into account essential aspects of the user's QoE discussed in this work. In addition, current work on DASH architectures for future generations of smart grids ignores the behavior of the video player in Smart City environments. ,  On the client side, aspects of the way in which the video player behaves to provide a delicate balance between cost and customer satisfaction (in terms of QoE), as well as the impact that the insertion of cross traffic can have on the viewer . While on the server side, new simulation scenarios with different domains between nodes in the mist need to be considered, as well as performing the communication between these domains in multilevel environments.

In this paper, the characteristics of a multi-tier edge/cloud scenarios with a VoD service is investigated. Numerical results for a binary tree network suggest that the correct video management can result in a substantial improvement in the users' QoE. However, introducing simple connection changes between the Aps and users, with the lack of an adequate orchestration of the connections, can negatively impact user satisfaction.

As future work, we intend to implement mechanisms capable of orchestrating the users' connections in real-time and improve the provisioning of video streaming in multi-tier edge/cloud environments.
Another improvement is assess how service behaves to provide a delicate balance between cost and customer satisfaction in terms of QoE.

%This work proposes a multi-tier architecture with a set of video-related services. The services were designed following the ETSI-NFV architecture. It focuses on the demonstration of the suitability of the services for multi-tier fog/cloud envi- ronments. In doing that, several properties of fog computing were further characterized. The proposed work combines recent fog/cloud technologies with state-of-the-art multi- tiered computing environments. This justifies the need for investigating specific services for video streaming provision- ing. As future work, we intend to implement the principles of NFV-SDN-based on
