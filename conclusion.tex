\section{Conclusion}
\label{sec:conclusion}

Through a literature review, it can be seen that there are many proposals in the area of ​​adaptive video streaming that do not take into account essential aspects of the user's QoE discussed in this work. In addition, current work on DASH architectures for future generations of smart grids ignores the behavior of the video player in Smart City environments. In the continuation of this work, we intend to work on the implementation of mechanisms to improve the provisioning of video streaming in multilevel architectures. On the client side, aspects of the way in which the video player behaves to provide a delicate balance between cost and customer satisfaction (in terms of QoE), as well as the impact that the insertion of cross traffic can have on the viewer . While on the server side, new simulation scenarios with different domains between nodes in the mist need to be considered, as well as performing the communication between these domains in multilevel environments.